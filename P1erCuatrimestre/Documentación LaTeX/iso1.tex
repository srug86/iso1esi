\documentclass[12pt,a4paper,spanish,twoside]{article}

% Español
\usepackage[spanish]{babel}
\usepackage{lmodern}
\usepackage[utf8]{inputenc}

% Imágenes
\usepackage[pdftex]{graphicx}
\usepackage{latexsym}
\usepackage{fancybox}
\usepackage{float}

% Ruta para las imágenes
\graphicspath{{imagesinterface/}}

% Colores
\usepackage{color}
\usepackage{colortbl}

% Margenes
\usepackage[margin=2.5cm,top=3cm]{geometry}

% Párrafos
\setlength{\parskip}{6pt}

% Entorno Listings para código fuente
\usepackage{listingsutf8}[2007/11/11]

\lstset{
  frame=Ltb, %forma del cuadro
  framerule=0pt, %ancho del borde
  aboveskip=0.5cm, %separación de los números de línea
  framexleftmargin=0.4cm, %margen externo izquierdo
  framesep=0pt,
  rulesep=.4pt,
  rulesepcolor=\color{black},
  % 
  stringstyle=\ttfamily,
  showstringspaces = false,
  basicstyle=\footnotesize,
  keywordstyle=\bfseries,
  % 
  numbers=left,
  numbersep=15pt,
  numberstyle=\tiny,
  numberfirstline= false,
  %
  inputencoding=utf8/latin1
  %texcl=true
}

% minimizar fragmentado de listados
\lstnewenvironment{listing}[1][]{
  \lstset{#1}\pagebreak[0]}{\pagebreak[0]
}

%
% \ifx\documentclass\undefinedcs
%      \def\bf{\fam\bffam\tenbf}\def\rm{\fam0\tenrm}\fi
%                          % f**k-amstex!
%

% Fancyhdr - Encabezados y pies de página
\usepackage{fancyhdr}
% Márgenes
\headsep=8mm
\footskip=14mm

% Fancy Header Style Options
\pagestyle{fancy}               % Sets fancy header and footer
\fancyfoot{}                    % Delete current footer settings

% Capítulo
% \renewcommand{\chaptermark}[1]{ % Lower Case Chapter marker style
%   \markboth{\chaptername\ \thechapter.\ #1}{}} 

% Sección
\renewcommand{\sectionmark}[1]{ % Lower case Section marker style
  \markright{\thesection.\ #1}} 

% Página
\fancyhead[LE,RO]{\bfseries\thepage} % Page number (boldface) in left on even
                                     % pages and right on odd pages

% ------------------ Macro para encabezados y pies de página-------------------
%    Uso: \encabezado{pares(pag izquierda)}
% -----------------------------------------------------------------------------
\def\encabezado#1{
  \fancyhead[RE]{\bfseries#1}     % In the right on even pages
  \fancyhead[LO]{\bfseries\rightmark}  % In the left on odd pages
  \renewcommand{\headrulewidth}{0.5pt} % Width of head rule
}
% -----------------------------------------------------------------------------


% ------------------ Macro para insertar una imagen ---------------------------
%    Uso: \imagen{nombreFichero}{Ancho}{Etiqueta}{Identificador}{Colocador}
% -----------------------------------------------------------------------------
\usepackage{ifthen}
\def\imagen#1#2#3#4#5{
%  \ifthenelse{\equal{#5}{}}{\begin{figure}[!h]}{\begin{figure}[#5]}
\begin{figure}[H]
    \begin{center}
      \ifthenelse{\equal{#2}{}}
      {\includegraphics{#1}}{\resizebox{#2}{!}{\includegraphics{#1}}}
      \ifthenelse{\equal{#3}{}}{}{\caption{#3}}
      \label{#4}
    \end{center}
  \end{figure}
}
% -----------------------------------------------------------------------------


% ------------------ Macro para la portada ------------------------------------
%    Uso: \portada{asignatura}{titulo}{subtítulo}{autor}{fecha}
% -----------------------------------------------------------------------------
\def\portada#1#2#3#4#5{
  \thispagestyle{empty}
  \vspace*{-2cm}

  \begin{center}
    \includegraphics[scale=0.25]{logoesi.pdf}
  
    \vspace*{1.5cm}
    {\Large \textbf{Universidad de Castilla-La Mancha\\ 
        Escuela Superior de Informática}}
    
    \vspace*{1.2cm}
    {\Large \textbf{#1}}
    
    \vspace*{1.5cm}
    {\huge #2}\\{\Large #3}
    
    \vspace*{1.5cm}
    {\large #4}
    \vfill
    \large{#5}
  \end{center}

  \newpage
  \vspace*{1cm}
  \thispagestyle{empty} 
  \newpage
}
% -----------------------------------------------------------------------------


% Margenes
\usepackage[margin=2.5cm,top=3cm]{geometry}
\begin{document}
\tableofcontents
\newpage
\section{Introducción}
\input{Introduccion}
\section{Especificación de Requisitos}
\subsection{Requisitos iniciales de Sistema}
Se trata de desarrollar una aplicación para gestión distribuída de la revisión de proyectos
de investigación (y otro tipo de solicitudes, como becas, acciones integradas, etc.). El
sistema lo mantiene una agencia de evaluación de proyectos, que básicamente se
encarga de ofrecer una valoración de los proyectos de investigación que le envían
distintos organismos (ministerios, comunidades autónomas, etc.). Existe un conjunto de
áreas temáticas, y cada área está descompuesta en un conjunto de subareas. Cada área
tiene una persona ``coordinadora'', que se encarga de asignar proyectos a cada uno de los
``adjunto'' de cada subarea. El adjunto se encargará de asignar la evaluación de sus
proyectos a los expertos más adecuados, y de, finalmente realizar los informes finales
de evaluación. El sistema es utilizado por los siguientes tipos de usuario:

\begin{itemize}
\item Los expertos, que realizan evaluaciones de proyectos. Se les invita, y ellos pueden
aceptar o declinar, y si aceptan tiene un tiempo específico para enviar sus informes.
\item Los adjuntos, que realizan asignaciones a expertos. También desasignan o insisten si el
experto tarda demasiado. Una vez recibidos los informes de los expertos, realiza un
único informe final, que es el que se devuelve a la entidad solicitante, una vez validado
por el coordinador del área correspondiente.
\item Los coordinadores, que asignan proyectos a los adjuntos, y realizan la supervisión de
todos los informes.
\item Secretario de la agencia de evaluación, que carga en el sistema todos los
documentos de los proyectos (memoria del proyecto, currículum de los
investigadores, texto de la convocatoria, etc.).
\end{itemize}

\subsection{Análisis de requisitos del sistema}
Tras una serie de reuniones, los requisitos finales para nuestro sistema son los siguientes:
\begin{itemize}

\item \textbf{Usuarios}
\begin{itemize}
\item Acceden al sistema mediante un nombre de usuario, que será la cuenta de correo; y una 
contraseña, que podrá modificar.
\item Hay cuatro tipos: secretario, coordinador, adjunto y experto. Cada uno con un diferente 
tipo de funcionalidad y rango.
\item Cada usuario podrá modificar sus datos personales y tendrá una vista restringida sobre la 
lista de proyectos dependiendo de su rango en el sistema.
\end{itemize}

\item \textbf{Paquete de Proyectos}
\\Las instituciones solicitantes mandan los proyectos en paquetes al secretario, los cuales 
contienen:
\begin{itemize}
\item La convocatoria.
\item Las bases del proyecto.
\item Institución convocante.
\item Los proyectos, que pueden venir, aunque no necesariamente, clasificados por área.
\item Uno o varios modelos de informe de evaluación que contengan los puntos a evaluar del 
correspondiente proyecto. 
\item Cada proyecto tendrá una fecha en la cuál tiene que estar evaluado.
\end{itemize}

\item \textbf{Secretario}
\begin{itemize}
\item Usuarios: Es el encargado de añadir, modificar y eliminar a los usuarios del sistema: 
coordinadores, adjuntos 
y expertos.
\item Coordinadores: Elegirá el coordinador de cada área.
\item Paquetes de proyecto: Recibe las solicitudes de evaluación de proyectos y los introduce 
al sistema: las bases, la convocatoria, la institución convocante, los proyectos, el cual asignará 
al área correspondiente; etc. También podrá modificar cualquier información referente a estas solicitudes. 
\item Modelos de evaluación: Dependiendo de la información que contenga el proyecto, elaborará unos 
modelos de informe de evaluación.
\item Plazos expertos: Decidirá los plazos que tienen los expertos para aceptar o declinar la invitación 
para realizar el informe de evaluación, así como una vez aceptado, la fecha para entregar dicho informe. 
Todo esto se incluirá en el modelo de evaluación.
\end{itemize}

\item \textbf{Coordinador}
\begin{itemize}
\item Pertenece a una única área.
\item Establece las subáreas de los proyectos que son asignados a su área.
\item Asigna al adjunto de cada subárea.
\item Reasignar proyecto a otra subárea, si el adjunto se lo indica.
\item Valida los informes pendientes que los adjuntos de las subáreas de su área realizan.
\end{itemize}

\item \textbf{Área}
\begin{itemize}
\item Está asociada a un único coordinador y tiene, a su vez, varias subáreas. 
\item El número de subáreas podrá ser diferente en cada área.
\end{itemize}

\item \textbf{Adjunto}
\begin{itemize}
\item Pertenece únicamente a una subárea.
\item Tendrá una cola de proyectos asignados.
\item Buscará a los expertos especificando el área, la institución, palabras clave y valoraciones 
en las que se prioriza la formalidad de plazos y calidad de las evaluaciones.
\item Una vez finalizada la búsqueda, elegirá a uno o más expertos según considere necesario y les enviará 
un modelo de invitación predeterminada mediante correo electrónico para la  evaluación del proyecto. 
\item Avisará de los plazos que tiene el experto para aceptar o declinar una invitación según definió el 
secretario. Además, una vez aceptada dicha invitación, le avisará de los plazos de entrega del informe 
de evaluación, también definidos por el secretario.
\item Podrá reasignar las evaluaciones si el experto declina la invitación, no obtiene contestación dentro 
del plazo o si el experto no cumple con los plazos de entrega del informe de evaluación.
\item El adjunto podrá insistir cuando esté próxima la fecha límite de entrega del informe de evaluación.
\item Una vez realizada las evaluaciones de los expertos, el adjunto realizará un informe final teniendo en cuenta 
los informes de los distintos expertos que hayan aceptado realizar la evaluación. Este informe final deberá ser 
validado por el coordinador de su área.
\item Evalúa el trabajo del experto basándose en la formalidad de los plazos y la calidad de su informe.
\item Puede recomendar al secretario añadir expertos.
\item Dentro de la lista de proyectos a las que los adjuntos tienen acceso, tendrán una sublista de los expertos 
que están revisando ese proyecto.
\item Podrá avisar al coordinador cuando el proyecto no corresponda a su subárea.
\end{itemize}

\item \textbf{Subárea}
\begin{itemize}
\item Sólo podrá pertenecer a un área y tiene un único adjunto asociado.
\end{itemize}

\item \textbf{Expertos}
\begin{itemize}
\item Podrán tener asignados varios proyectos a la vez. 
\item No podrán pertenecer a la misma institución solicitante de la evaluación.
\item Podrá aceptar o declinar las invitaciones de evaluación de proyectos.
\item Cuando acepte la invitación, podrá acceder a la documentación de ese proyecto e ir realizando 
progresivamente el informe en varias sesiones.
\item Una vez que haya terminado el informe finalizará el proceso de evaluación.
\item Recibirá avisos de finalización de plazos por parte del adjunto para finalizar el informe, 
vía correo electrónico.
\item Cada uno tendrá una serie de palabras clave asociadas a su temática. Éstas palabras clave se 
utilizarán como parámetros en las búsquedas.
\item Tendrá una lista de evaluaciones pendientes, que podrá aceptar o rechazar.
\end{itemize}

\end{itemize}

\section{Análisis de sistema}
\subsection{Diagrama de flujo de datos}

\imagen{diagramas/DC.pdf}{15.5cm}{Diagrama de Contexto}{dc}
\imagen{diagramas/DFD0.pdf}{15.5cm}{Diagrama de Sistemas}{0}
\imagen{diagramas/DFD1.pdf}{15.5cm}{Gestión de Usuarios}{1}
\imagen{diagramas/DFD2.pdf}{15.5cm}{Gestión de Paquetes Proyectos}{2}
\imagen{diagramas/DFD3.pdf}{15.5cm}{Gestión de Evaluaciones}{3}
\imagen{diagramas/DFD31.pdf}{15.5cm}{Gestión de Modelos}{31}
\imagen{diagramas/DFD32.pdf}{15.5cm}{Gestión de Informes Evaluaciones}{32}
\imagen{diagramas/DFD4.pdf}{15.5cm}{Gestión de Expertos}{4}

\subsection{Definición de Flujo de Datos}
\input{diccionario}

\section{Diseño del Sistema}
\subsection{Diagrama de Diseño Estructurado}
\imagen{diagramas/DE_1.pdf}{15cm}{\small{Gestión distribuida de la revisión de Proyectos de Investigación}}{}
\imagen{diagramas/DE_2.pdf}{15cm}{\small{Gestión de Usuarios}}{}
\imagen{diagramas/DE_3.pdf}{15cm}{\small{Gestión de Paquetes de Proyectos}}{}
\imagen{diagramas/DE_4.pdf}{15cm}{\small{Gestión de Evaluaciones}}{}
\imagen{diagramas/DE_5.pdf}{15cm}{\small{Gestión de modelos}}{}
\imagen{diagramas/DE_6.pdf}{14cm}{\small{Gestión de Informes de Evaluaciones}}{}
\imagen{diagramas/DE_7.pdf}{14cm}{\small{Gestión de Expertos}}{}

\section{Base de Datos}
\subsection{Diagrama Entidad/Interrelación}
%\imagen{diagramas/ER.pdf}{15cm}{Diagrama Entidad/Interrelación}{}
\subsection{Diagrama Relacional}
%\imagen{diagramas/MRE.pdf}{15cm}{Diagrama Relacional}{}

\section{Manual de usuario}
\input{manual}
\section{Implementación}
\subsection{Introducción}
\subsection{Código fuente}
\subsubsection{index.php}
\lstinputlisting{implementacion/index.php}
\subsubsection{settings.php}
\lstinputlisting{implementacion/settings.php}
\subsubsection{evalmodels.js}
\lstinputlisting{implementacion/js/evalmodels.js}
\subsubsection{evalreports.js}
\lstinputlisting{implementacion/js/evalreports.js}
\subsubsection{main.js}
\lstinputlisting{implementacion/js/main.js}
\subsubsection{projects.js}
\lstinputlisting{implementacion/js/projects.js}

\subsubsection{evaluation\_models.php}
\lstinputlisting{implementacion/source/evaluation_models.php}
\subsubsection{evaluation\_reports.php}
\lstinputlisting{implementacion/source/evaluation_reports.php}
\subsubsection{loginout.php}
\lstinputlisting{implementacion/source/loginout.php}
\subsubsection{mysql.php}
\lstinputlisting{implementacion/source/mysql.php}

\subsubsection{evaluation\_models.php}
\lstinputlisting{implementacion/theme/evaluation_models.php}
\subsubsection{evaluation\_reports.php}
\lstinputlisting{implementacion/theme/evaluation_reports.php}
\subsubsection{login.php}
\lstinputlisting{implementacion/theme/login.php}
\subsubsection{main.php}
\lstinputlisting{implementacion/theme/main.php}
\subsubsection{projects.php}
\lstinputlisting{implementacion/theme/projects.php}
\subsubsection{users.php}
\lstinputlisting{implementacion/theme/users.php}

\subsubsection{evalmodels.css}
\lstinputlisting{implementacion/theme/css/evalmodels.css}
\subsubsection{evalreports.css}
\lstinputlisting{implementacion/theme/css/evalreports.css}
\subsubsection{login.css}
\lstinputlisting{implementacion/theme/css/login.css}
\subsubsection{main.css}
\lstinputlisting{implementacion/theme/css/main.css}
\subsubsection{users.php}
\lstinputlisting{implementacion/theme/css/users.php}

\section{Bibliografía}

\end{document}

