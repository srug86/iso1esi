% Clase
\documentclass[11pt,a4paper,spanish,twoside]{book}

% Órdenes auxiliares
\input{inc/includes.tex}

% Encabezado y pie de página
\encabezado

\begin{document}

% Silabación extra
\hyphenation{
a-sig-na-tu-ras
au-to-ma-ti-za-rá
ca-li-dad
ca-tá-lo-go
ca-rre-ra
cons-truc-ción
co-rrec-ta-men-te
co-rres-pon-de
de-pen-dien-tes
de-sa-rro-llo
des-tru-yen
diag-nos-tico
dis-po-ni-bi-li-dad
e-va-lua-ción
fi-na-li-za-ción
ge-ne-ra-ción
in-fe-rior
man-te-ni-mien-to
me-ca-nis-mo
me-dian-te
o-cu-rren-cia
per-so-nal
pos-te-rio-res
pro-ce-di-mien-to
pro-ce-di-mien-tos
pro-por-cio-na-rá
pu-bli-ca-da
re-a-li-za-da
re-cha-zar
re-fe-ren-cia
re-lle-nar
re-qui-si-tos
res-pec-ti-va-men-te
res-pecto
res-pon-sa-bi-li-dad
se-cre-ta-rio
u-su-a-rios
va-ria-rá
vi-lla-rre-al
}


% Portada
\portada{Ingeniería del Software}
{Práctica 2}{Diseño orientado a objetos}
{Sergio de la Rubia García-Carpintero\\Miguel Millán Sánchez-Grande\\
  Luis Muñoz Villarreal\\Alicia Serrano Sánchez\\
  Juan Miguel Torres Triviño}{28 de Mayo de 2010}

% Licencia
\licencia{Sergio de la Rubia García-Carpintero, Miguel Millán Sánchez-Grande,
  Luis Muñoz Villarreal, Alicia Serrano Sánchez, Juan Miguel Torres Triviño}

% Índices
\tableofcontents
%\listoftables
%\listoffigures

\chapter*{Introducción}
El objetivo de la práctica es conocer y simular el ciclo de vida seguido
durante el desarrollo del software. Para ello se va a realizar un supuesto
práctico que incluye análisis de requisitos, diseño e implementación del
mismo. 

Dado que el enunciado fijado para la práctica no proporcionaba toda la
información necesaria para poder desarrollarla, se han realizado
una serie de entrevistas con el cliente para poder afianzar los requisitos
finales de la aplicación. Éstas reuniones son fundamentales para poder
conseguir un buen análisis e ir profundizando sobre las diferencias de las
ideas del cliente y las del equipo de síntesis y desarrollo del software, lo
que hace un paso imprescindible para un buen diseño y una buena implementación.

Además, al tratarse de un proyecto en grupo, proporciona habilidades en
nuestra formación como ingenieros al tener que enfrentarse a un importante
aspecto de la vida real, el trabajo en equipo. Esto tiene partes positivas y
negativas, ya que al ser varios miembros se han de poner en común distintos
puntos de vista y llegar a unacuerdo para poder proseguir con el desarrollo;
pero también agiliza las etapas, ya que se pueden repartir las diferentes
tareas a realizar del trabajo. 

Con la realización de esta práctica se pretende conocer y asimilar los
objetivos básicos de la ingeniería del software: 
\begin{itemize}
\item Los procesos del ciclo de vida software y sus diferentes formas de
  organización en distintos modelos del ciclo de vida.
\item Los conceptos y actividades fundamentales del análisis de requisitos,
  así como su importancia en el desarrollo y mantenimiento del software.
\item Los conceptos, técnicas y diagramas básicos del paradigma de desarrollo
  orientado a objetos: desde el análisis a las pruebas.
\item Un modelo de proceso de aplicación del paradigma orientado a objetos, que
  incluya el proceso de análisis, diseño y estrategias de prueba.
\item Las posibilidades que ofrece la reutilización del software en todos los
  niveles de desarrollo.
\end{itemize}
\chapter{Especificación de requisitos}

\section{Requisitos iniciales del sistema}
\section{Análisis de requisitos del sistema}

\section{Análisis de requisitos}
\subsection{Usuarios}

\subsection{Paquete de proyectos}

\subsection{Secretario}

\subsection{Coordinador}

\subsection{Área}

\subsection{Adjunto}

\subsection{Subárea}

\subsection{Expertos}

\chapter{Análisis}

\section{Diagrama de casos de uso}

\section{Diagramas de secuencia}

\subsection{Entrada al sistema correcta}

\subsection{Entrada al sistema errónea}

\subsection{Secretario introduce paquete de proyectos}

\subsection{Coordinador selecciona experto para ser adjunto}

\subsection{Coordinador asigna un proyecto a una subárea}

\subsection{Coordinador valida informe}

\subsection{Adjunto busca expertos para evaluar proyecto}

\subsection{Adjunto realiza informe}

\subsection{Experto realiza informe}

\section{Diagrama Entidad/Interrelación}
El modelo entidad/interrelación es el modelo conceptual más utilizado para el 
diseño conceptual de bases de datos.

Es un modelo de red que describe la distribución de los datos almacenados en un 
sistema de forma abstracta. La abstracción busca las propiedades comunes de un 
conjunto de objetos reduciendo la complejidad y ayudando a entender el mundo 
real. Se pretende \emph{visualizar} los objetos que pertenecen a la
\emph{base de datos} como entidades que tienen unos atributos y se vinculan
mediante relaciones.

\imagen{ER.pdf}{14.5}{Diagrama Entidad/Interrelación}{}{}

\chapter{Diseño}

\section{Diagrama de clases}

\section{Diagrama relacional}
Es el modelo más utilizado en la actualidad para modelar problemas reales y 
administrar datos dinámicamente.

Se trata de un modelo lógico, que establece una estructura sobre los datos,
aunque posteriormente éstos puedan ser almacenados de múltiples formas para
aprovechar características físicas concretas de la máquina sobre la que se
implante la base de datos realmente.

\imagen{MRE.pdf}{15}{Diagrama Relacional}{}{}

\chapter{Implementación}

\section{Manual de usuario}

\section{Implementación}

\appendix
\chapter{Carga de trabajo}
\begin{center}
  \begin{tabular}{p{10cm}|c}
    \textbf{Apellidos y Nombre} & \textbf{Porcentaje} \\ \hline \hline
    de la Rubia García-Carpintero, Sergio & 20\% \\
    Millán Sánchez-Grande, Miguel         & 20\% \\ 
    Muñoz Villarreal, Luis                & 20\% \\ 
    Serrano Sánchez, Alicia               & 20\% \\ 
    Torres Triviño, Juan Miguel           & 20\% \\
  \end{tabular}
\end{center}

\end{document}
