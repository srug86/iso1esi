% Clase
\documentclass[11pt,a4paper,spanish,twoside]{book}

% Órdenes auxiliares
\input{inc/includes.tex}

% Encabezado y pie de página
\encabezado

\begin{document}

% Silabación extra
\hyphenation{
a-sig-na-tu-ras
au-to-ma-ti-za-rá
ca-li-dad
ca-tá-lo-go
ca-rre-ra
cons-truc-ción
co-rrec-ta-men-te
co-rres-pon-de
de-pen-dien-tes
de-sa-rro-llo
des-tru-yen
diag-nos-tico
dis-po-ni-bi-li-dad
e-va-lua-ción
fi-na-li-za-ción
ge-ne-ra-ción
in-fe-rior
man-te-ni-mien-to
me-ca-nis-mo
me-dian-te
o-cu-rren-cia
per-so-nal
pos-te-rio-res
pro-ce-di-mien-to
pro-ce-di-mien-tos
pro-por-cio-na-rá
pu-bli-ca-da
re-a-li-za-da
re-cha-zar
re-fe-ren-cia
re-lle-nar
re-qui-si-tos
res-pec-ti-va-men-te
res-pecto
res-pon-sa-bi-li-dad
se-cre-ta-rio
u-su-a-rios
va-ria-rá
vi-lla-rre-al
}


% Portada
\portada{Ingeniería del Software}
{Práctica 2}{Diseño orientado a objetos}
{Sergio de la Rubia García-Carpintero\\Miguel Millán Sánchez-Grande\\
  Luis Muñoz Villarreal\\Alicia Serrano Sánchez\\
  Juan Miguel Torres Triviño}{28 de Mayo de 2010}

% Licencia
\licencia{Sergio de la Rubia García-Carpintero, Miguel Millán Sánchez-Grande,
  Luis Muñoz Villarreal, Alicia Serrano Sánchez, Juan Miguel Torres Triviño}

% Índices
\tableofcontents
%\listoftables
%\listoffigures

\chapter*{Introducción}
El objetivo de la práctica es conocer y simular el ciclo de vida seguido
durante el desarrollo del software. Para ello se va a realizar un supuesto
práctico que incluye análisis de requisitos, diseño e implementación del
mismo. 

Dado que el enunciado fijado para la práctica no proporcionaba toda la
información necesaria para poder desarrollarla, se han realizado
una serie de entrevistas con el cliente para poder afianzar los requisitos
finales de la aplicación. Éstas reuniones son fundamentales para poder
conseguir un buen análisis e ir profundizando sobre las diferencias de las
ideas del cliente y las del equipo de síntesis y desarrollo del software, lo
que hace un paso imprescindible para un buen diseño y una buena implementación.

Además, al tratarse de un proyecto en grupo, proporciona habilidades en
nuestra formación como ingenieros al tener que enfrentarse a un importante
aspecto de la vida real, el trabajo en equipo. Esto tiene partes positivas y
negativas, ya que al ser varios miembros se han de poner en común distintos
puntos de vista y llegar a unacuerdo para poder proseguir con el desarrollo;
pero también agiliza las etapas, ya que se pueden repartir las diferentes
tareas a realizar del trabajo. 

Con la realización de esta práctica se pretende conocer y asimilar los
objetivos básicos de la ingeniería del software: 
\begin{itemize}
\item Los procesos del ciclo de vida software y sus diferentes formas de
  organización en distintos modelos del ciclo de vida.
\item Los conceptos y actividades fundamentales del análisis de requisitos,
  así como su importancia en el desarrollo y mantenimiento del software.
\item Los conceptos, técnicas y diagramas básicos del paradigma de desarrollo
  orientado a objetos: desde el análisis a las pruebas.
\item Un modelo de proceso de aplicación del paradigma orientado a objetos, que
  incluya el proceso de análisis, diseño y estrategias de prueba.
\item Las posibilidades que ofrece la reutilización del software en todos los
  niveles de desarrollo.
\end{itemize}
\chapter{Especificación de requisitos}
\section{Requisitos iniciales del sistema}
Se desarrolla una aplicación para la gestión distribuída de la
revisión de proyectos de investigación, además de prestar soporte para
distintas solicitudes (becas, acciones integradas, etc.). El sistema lo
mantiene una agencia de evaluación de proyectos, que es la encargada
de ofrecer una valoración de los proyectos de investigación que le envían
distintos organismos (ministerios, comunidades autónomas, etc.). Existe un
conjunto de áreas temáticas y cada área está compuesta por un conjunto de
subáreas. Cada área tiene un \emph{coordinador}, que se encarga de asignar
proyectos a cada uno de los \emph{adjuntos} de cada subárea. El
\emph{adjunto} se encargará de asignar la evaluación de sus proyectos a los
expertos más adecuados y de, finalmente, realizar los informes finales de
evaluación. El sistema es utilizado por los siguientes tipos de usuario: 

\begin{itemize}
\item Los \emph{expertos} realizan evaluaciones de proyectos. Reciben una 
  invitación que pueden aceptar o declinar. Una vez que los expertos aceptan
  la invitación, tendrán un tiempo determinado para enviar sus informes. 
\item Los \emph{adjuntos} realizan asignaciones de proyectos a
  \emph{expertos}. También tienen la funcionalidad de desasignar a
  \emph{expertos} o insistir si el experto tarda demasiado en la realización
  de un informe. Una vez recibidos los informes de los \emph{expertos}, el
  adjunto realiza un único informe final, que se devuelve a la entidad
  solicitante una vez validado por el \emph{coordinador} del área
  correspondiente.     
\item Los \emph{coordinadores} asignan proyectos a los \emph{adjuntos} y
  realizan la supervisión de todos los informes. 
\item El \emph{secretario} de la agencia de evaluación carga en el sistema todos
  los documentos de los proyectos (memoria del proyecto, currículum de los
  investigadores, texto de la convocatoria, etc.). 
\end{itemize}

\section{Análisis de requisitos del sistema}
Tras una serie de reuniones, los requisitos finales para nuestro sistema son
los siguientes: 

\subsection{Usuarios}
\begin{itemize}
\item Acceden al sistema mediante un identificador, que será el dni; y una 
contraseña, que se podrá modificar.
\item Hay cuatro tipos: secretario, coordinador, adjunto y experto. Cada uno
  con diferente funcionalidad y rango. 
\item Cada usuario podrá modificar sus datos personales y tendrá una vista
  restringida sobre la lista de proyectos dependiendo de su rango en el 
  sistema.
\end{itemize}

\subsection{Paquete de proyectos}
Las instituciones solicitantes mandan los proyectos en paquetes al
secretario, los cuales contienen: 
\begin{itemize}
\item La convocatoria, que describe uno o varios modelos de informe de
  evaluación que contienen los criterios de evaluación del correspondiente
  proyecto. 
\item Las bases del proyecto.
\item La institución convocante.
\item Los proyectos, que opcionalmente pueden venir clasificados por área. 
\item Cada proyecto tendrá un plazo estipulado para que la evaluación esté
  realizada. 
\end{itemize}

\subsection{Secretario}
\begin{description}
\item[Usuarios] Es el encargado de añadir, modificar y eliminar 
  usuarios del sistema (coordinadores, adjuntos y expertos).
\item[Coordinadores] Elegirá el coordinador de cada área.
\item[Paquetes de proyecto] Recibe las solicitudes de evaluación de proyectos
  en forma de paquetes de proyectos y los introduce al sistema (las bases, la
  convocatoria, la institución convocante, los proyectos, el cual asignará al
  área correspondiente...). También podrá modificar cualquier información
  referente a estas solicitudes.  
\item[Modelos de evaluación] Dependiendo de la información que contenga el
  proyecto, elaborará unos modelos de informe de evaluación. 
\item[Plazos expertos] Decidirá los plazos que tienen los expertos para
  aceptar o declinar la invitación para realizar el informe de evaluación,
  y una vez aceptada, la fecha para entregar dicho informe.
\end{description}

\subsection{Coordinador}
\begin{itemize}
\item Pertenece a una única área.
\item Establece las subáreas de los proyectos que se le asignan.
\item Dentro de su área asigna un adjunto a cada subárea.
\item Reasigna un proyecto a otra subárea, si el adjunto se lo indica.
\item Valida los informes pendientes que los adjuntos de su correspondiente 
  área realizan.
\item Puede asignar a un experto para ser adjunto de una subárea.
\end{itemize}

\subsection{Área}
\begin{itemize}
\item Está asociada a un único coordinador y tiene, a su vez, varias subáreas. 
\item El número de subáreas podrá ser diferente en cada área.
\end{itemize}

\subsection{Adjunto}
\begin{itemize}
\item Pertenece únicamente a una subárea.
\item Tiene una serie de proyectos asignados.
\item Busca a los expertos especificando el área, la institución, palabras
  clave y valoraciones en las que se prioriza la formalidad de plazos y
  calidad de las evaluaciones. 
\item Una vez finalizada la búsqueda, elige a uno o más expertos según
  considere necesario y les envía un modelo de invitación predeterminada
  mediante correo electrónico para la evaluación del proyecto.  
\item Puede reasignar las evaluaciones si el experto declina la invitación,
  no obtiene contestación dentro del plazo o si el experto no cumple con los
  plazos de entrega del informe de evaluación.
\item El adjunto puede insistir a los expertos correspondientes cuando esté
  próxima la fecha límite de entrega del informe de evaluación.
\item Una vez realizada las evaluaciones de los expertos, el adjunto
  realiza un informe final teniendo en cuenta los informes de los distintos
  expertos que hayan aceptado realizar la evaluación. Este informe final
  debe ser validado por el coordinador de su área. 
\item Evalúa el trabajo del experto basándose en la formalidad de los plazos
  y la calidad de su informe. 
\item Puede recomendar al secretario añadir expertos.
\item Dentro de la lista de proyectos a las que los adjuntos tienen acceso,
  tiene una sublista de los expertos que están realizando la evaluación de
  ese proyecto.  
\item Puede avisar al coordinador cuando el proyecto no corresponda a su
  subárea. 
\end{itemize}

\subsection{Subárea}
\begin{itemize}
\item Sólo puede pertenecer a un área y tiene un único adjunto asociado.
\end{itemize}

\subsection{Expertos}
\begin{itemize}
\item Pueden tener asignados varios proyectos a la vez. 
\item No pueden pertenecer a la misma institución solicitante de la evaluación.
\item Pueden aceptar o declinar las invitaciones de evaluación de proyectos.
\item Cuando acepte la invitación, pueden acceder a la documentación de ese
  proyecto e ir realizando progresivamente el informe en varias sesiones. 
\item Una vez que terminen el informe finalizan el proceso de evaluación.
\item Reciben avisos de finalización de plazos por parte del adjunto para
  finalizar el informe, vía correo electrónico.
\item Cada uno tiene una serie de palabras clave asociadas a su
  temática. Éstas palabras clave se utilizan como parámetros en las
  búsquedas. 
\item Tienen una lista de evaluaciones pendientes, que pueden aceptar o 
  rechazar.
\item El sistema avisa de los plazos que tiene el experto para aceptar o
  declinar una invitación según definió el secretario. Además, una vez
  aceptada dicha invitación, se le comunica los plazos de entrega del informe
  de evaluación, también definidos por el secretario.   

\end{itemize}


\chapter{Análisis}

\section{Diagrama de casos de uso}

\section{Diagramas de secuencia}

\subsection{Entrada al sistema correcta}

\subsection{Entrada al sistema errónea}

\subsection{Secretario introduce paquete de proyectos}

\subsection{Coordinador selecciona experto para ser adjunto}

\subsection{Coordinador asigna un proyecto a una subárea}

\subsection{Coordinador valida informe}

\subsection{Adjunto busca expertos para evaluar proyecto}

\subsection{Adjunto realiza informe}

\subsection{Experto realiza informe}

\section{Diagrama Entidad/Interrelación}
El modelo entidad/interrelación es el modelo conceptual más utilizado para el 
diseño conceptual de bases de datos.

Es un modelo de red que describe la distribución de los datos almacenados en un 
sistema de forma abstracta. La abstracción busca las propiedades comunes de un 
conjunto de objetos reduciendo la complejidad y ayudando a entender el mundo 
real. Se pretende \emph{visualizar} los objetos que pertenecen a la
\emph{base de datos} como entidades que tienen unos atributos y se vinculan
mediante relaciones.

\imagen{ER.pdf}{14.5}{Diagrama Entidad/Interrelación}{}{}

\chapter{Diseño}

\section{Diagrama de clases}

\section{Diagrama relacional}
Es el modelo más utilizado en la actualidad para modelar problemas reales y 
administrar datos dinámicamente.

Se trata de un modelo lógico, que establece una estructura sobre los datos,
aunque posteriormente éstos puedan ser almacenados de múltiples formas para
aprovechar características físicas concretas de la máquina sobre la que se
implante la base de datos realmente.

\imagen{MRE.pdf}{15}{Diagrama Relacional}{}{}

\chapter{Implementación}

\section{Manual de usuario}

\section{Implementación}

\appendix
\chapter{Carga de trabajo}
\begin{center}
  \begin{tabular}{p{10cm}|c}
    \textbf{Apellidos y Nombre} & \textbf{Porcentaje} \\ \hline \hline
    de la Rubia García-Carpintero, Sergio & 20\% \\
    Millán Sánchez-Grande, Miguel         & 20\% \\ 
    Muñoz Villarreal, Luis                & 20\% \\ 
    Serrano Sánchez, Alicia               & 20\% \\ 
    Torres Triviño, Juan Miguel           & 20\% \\
  \end{tabular}
\end{center}

\end{document}
