\documentclass[12pt,a4paper,spanish,twoside]{article}

% Español
\usepackage[spanish]{babel}
\usepackage{lmodern}
\usepackage[utf8]{inputenc}

% Imágenes
\usepackage[pdftex]{graphicx}
\usepackage{latexsym}
\usepackage{fancybox}
\usepackage{float}

% Ruta para las imágenes
\graphicspath{{imagesinterface/}}

% Colores
\usepackage{color}
\usepackage{colortbl}

% Margenes
\usepackage[margin=2.5cm,top=3cm]{geometry}

% Párrafos
\setlength{\parskip}{6pt}

% Entorno Listings para código fuente
\usepackage{listingsutf8}[2007/11/11]

\lstset{
  frame=Ltb, %forma del cuadro
  framerule=0pt, %ancho del borde
  aboveskip=0.5cm, %separación de los números de línea
  framexleftmargin=0.4cm, %margen externo izquierdo
  framesep=0pt,
  rulesep=.4pt,
  rulesepcolor=\color{black},
  % 
  stringstyle=\ttfamily,
  showstringspaces = false,
  basicstyle=\footnotesize,
  keywordstyle=\bfseries,
  % 
  numbers=left,
  numbersep=15pt,
  numberstyle=\tiny,
  numberfirstline= false,
  %
  inputencoding=utf8/latin1
  %texcl=true
}

% minimizar fragmentado de listados
\lstnewenvironment{listing}[1][]{
  \lstset{#1}\pagebreak[0]}{\pagebreak[0]
}

%
% \ifx\documentclass\undefinedcs
%      \def\bf{\fam\bffam\tenbf}\def\rm{\fam0\tenrm}\fi
%                          % f**k-amstex!
%

% Fancyhdr - Encabezados y pies de página
\usepackage{fancyhdr}
% Márgenes
\headsep=8mm
\footskip=14mm

% Fancy Header Style Options
\pagestyle{fancy}               % Sets fancy header and footer
\fancyfoot{}                    % Delete current footer settings

% Capítulo
% \renewcommand{\chaptermark}[1]{ % Lower Case Chapter marker style
%   \markboth{\chaptername\ \thechapter.\ #1}{}} 

% Sección
\renewcommand{\sectionmark}[1]{ % Lower case Section marker style
  \markright{\thesection.\ #1}} 

% Página
\fancyhead[LE,RO]{\bfseries\thepage} % Page number (boldface) in left on even
                                     % pages and right on odd pages

% ------------------ Macro para encabezados y pies de página-------------------
%    Uso: \encabezado{pares(pag izquierda)}
% -----------------------------------------------------------------------------
\def\encabezado#1{
  \fancyhead[RE]{\bfseries#1}     % In the right on even pages
  \fancyhead[LO]{\bfseries\rightmark}  % In the left on odd pages
  \renewcommand{\headrulewidth}{0.5pt} % Width of head rule
}
% -----------------------------------------------------------------------------


% ------------------ Macro para insertar una imagen ---------------------------
%    Uso: \imagen{nombreFichero}{Ancho}{Etiqueta}{Identificador}{Colocador}
% -----------------------------------------------------------------------------
\usepackage{ifthen}
\def\imagen#1#2#3#4#5{
%  \ifthenelse{\equal{#5}{}}{\begin{figure}[!h]}{\begin{figure}[#5]}
\begin{figure}[H]
    \begin{center}
      \ifthenelse{\equal{#2}{}}
      {\includegraphics{#1}}{\resizebox{#2}{!}{\includegraphics{#1}}}
      \ifthenelse{\equal{#3}{}}{}{\caption{#3}}
      \label{#4}
    \end{center}
  \end{figure}
}
% -----------------------------------------------------------------------------


% ------------------ Macro para la portada ------------------------------------
%    Uso: \portada{asignatura}{titulo}{subtítulo}{autor}{fecha}
% -----------------------------------------------------------------------------
\def\portada#1#2#3#4#5{
  \thispagestyle{empty}
  \vspace*{-2cm}

  \begin{center}
    \includegraphics[scale=0.25]{logoesi.pdf}
  
    \vspace*{1.5cm}
    {\Large \textbf{Universidad de Castilla-La Mancha\\ 
        Escuela Superior de Informática}}
    
    \vspace*{1.2cm}
    {\Large \textbf{#1}}
    
    \vspace*{1.5cm}
    {\huge #2}\\{\Large #3}
    
    \vspace*{1.5cm}
    {\large #4}
    \vfill
    \large{#5}
  \end{center}

  \newpage
  \vspace*{1cm}
  \thispagestyle{empty} 
  \newpage
}
% -----------------------------------------------------------------------------


% Margenes
\usepackage[margin=2.5cm,top=3cm]{geometry}
\begin{document}
\portada{Ingeniería del Software I}{Práctica Primer Cuatrimestre}{Gestión 
Distribuída de la Revisión de Proyectos de Investigación}{Sergio de la Rubia
García-Carpintero\\Miguel Millán Sánchez-Grande\\Luis Muñoz Villarreal\\Alicia
Serrano Sánchez\\Juan Miguel Torres Triviño}{17 de Febrero del 2010}

\tableofcontents
\newpage
\section{Introducción}
El objetivo de la práctica es conocer y simular el ciclo de vida seguido
durante el desarrollo del software. Para ello se va a realizar un supuesto
práctico que incluye análisis de requisitos, diseño e implementación del
mismo. 

Dado que el enunciado fijado para la práctica no proporcionaba toda la
información necesaria para poder desarrollarla, se han tenido que realizar
sucesivas entrevistas con el cliente para poder afianzar los requisitos finales
de la aplicación. Éstas reuniones son fundamentales para poder conseguir un
buen análisis e ir profundizando en las diferencias entre las ideas del
cliente y las del equipo de síntesis y desarrollo del software, lo que hace
un paso imprescindible para un buen diseño y una buena
implementación. Además, al tratarse de un proyecto en grupo, nos
proporcionará habilidades en nuestra formación como ingenieros al tener que
enfrentarnos a un importante aspecto de la vida real, el trabajo en
equipo. Esto tiene partes positivas y negativas, ya que al ser varios
miembros se han de poner en común distintos puntos de vista y llegar a un
acuerdo para poder proseguir con el desarrollo; pero también agiliza las
etapas, ya que se pueden repartir las diferentes tareas a realizar del
trabajo. 

Con la realización de esta práctica se pretende conocer y asimilar los
objetivos básicos de la ingeniería del software: los procesos del ciclo de
vida software y sus diferentes formas de organización en distintos modelos
del ciclo de vida; los conceptos y actividades fundamentales del análisis
de requisitos y ser conscientes de la importancia que el análisis de
requisitos tiene en el desarrollo y mantenimiento del software; los conceptos,
técnicas y diagramas básicos del paradigma de desarrollo estructurado: desde
el análisis a las pruebas y el despliegue; un modelo de proceso de aplicación
del paradigma estructurado, que incluya el proceso de análisis y diseño
estructurado, heurísticas de transición entre ambos, y estrategias de prueba;
y las posibilidades que ofrece la reutilización del software en todos los
niveles de desarrollo. 

\section{Especificación de requisitos}
\subsection{Requisitos iniciales de sistema}
Se trata de desarrollar una aplicación para la gestión distribuída de la
revisión de proyectos de investigación (y otro tipo de solicitudes, como
becas, acciones integradas, etc.). El sistema lo mantiene una agencia de
evaluación de proyectos, que básicamente se encarga de ofrecer una valoración
de los proyectos de investigación que le envían distintos organismos
(ministerios, comunidades autónomas, etc.). Existe un conjunto de áreas
temáticas, y cada área está descompuesta en un conjunto de subáreas. Cada
área tiene una persona ``coordinadora'', que se encarga de asignar proyectos
a cada uno de los ``adjuntos'' de cada subárea. El adjunto se encargará de
asignar la evaluación de sus proyectos a los expertos más adecuados y de,
finalmente, realizar los informes finales de evaluación. El sistema es
utilizado por los siguientes tipos de usuario: 

\begin{itemize}
\item Los expertos, que realizan evaluaciones de proyectos. Reciben una 
  invitación, y ellos pueden aceptar o declinar, y si aceptan tiene un tiempo 
  específico para enviar sus informes. 
\item Los adjuntos, que realizan asignaciones a expertos. También desasignan
  expertos o insisten si el experto tarda demasiado. Una vez recibidos los 
  informes de los expertos, realiza un único informe final, que es el que se 
  devuelve a la entidad solicitante, una vez validado por el coordinador del 
  área correspondiente. 
\item Los coordinadores, que asignan proyectos a los adjuntos, y realizan la
  supervisión de todos los informes.
\item Secretario de la agencia de evaluación, que carga en el sistema todos
  los documentos de los proyectos (memoria del proyecto, currículum de los
  investigadores, texto de la convocatoria, etc.). 
\end{itemize}

\subsection{Análisis de requisitos del sistema}
Tras una serie de reuniones, los requisitos finales para nuestro sistema son
los siguientes: 

\subsubsection{Usuarios}
\begin{itemize}
\item Acceden al sistema mediante un nombre de usuario, que será la cuenta de
  correo; y una contraseña, que se podrá modificar.
\item Hay cuatro tipos: secretario, coordinador, adjunto y experto. Cada uno
  con un diferente tipo de funcionalidad y rango. 
\item Cada usuario podrá modificar sus datos personales y tendrá una vista
  restringida sobre la lista de proyectos dependiendo de su rango en el sistema.
\end{itemize}

\subsubsection{Paquete de proyectos}
Las instituciones solicitantes mandan los proyectos en paquetes al
secretario, los cuales contienen: 
\begin{itemize}
\item La convocatoria.
\item Las bases del proyecto.
\item Institución convocante.
\item Los proyectos, que pueden venir, aunque no necesariamente, clasificados
  por área. 
\item Uno o varios modelos de informe de evaluación que contengan los puntos
  a evaluar del correspondiente proyecto. 
\item Cada proyecto tendrá una fecha en la cuál tiene que estar evaluado.
\end{itemize}

\subsubsection{Secretario}
\begin{itemize}
\item Usuarios: Es el encargado de añadir, modificar y eliminar a los
  usuarios del sistema: coordinadores, adjuntos y expertos.
\item Coordinadores: Elegirá el coordinador de cada área.
\item Paquetes de proyecto: Recibe las solicitudes de evaluación de proyectos
  y los introduce al sistema: las bases, la convocatoria, la institución
  convocante, los proyectos, el cual asignará al área correspondiente;
  etc. También podrá modificar cualquier información referente a estas
  solicitudes. 
\item Modelos de evaluación: Dependiendo de la información que contenga el
  proyecto, elaborará unos modelos de informe de evaluación. 
\item Plazos expertos: Decidirá los plazos que tienen los expertos para
  aceptar o declinar la invitación para realizar el informe de evaluación,
  así como una vez aceptado, la fecha para entregar dicho informe. Todo esto
  se incluirá en el modelo de evaluación.
\end{itemize}

\subsubsection{Coordinador}
\begin{itemize}
\item Pertenece a una única área.
\item Establece las subáreas de los proyectos que son asignados a su área.
\item Asigna al adjunto de cada subárea.
\item Reasignar proyecto a otra subárea, si el adjunto se lo indica.
\item Valida los informes pendientes que los adjuntos de las subáreas de su
  área realizan.
\end{itemize}

\subsubsection{Área}
\begin{itemize}
\item Está asociada a un único coordinador y tiene, a su vez, varias subáreas. 
\item El número de subáreas podrá ser diferente en cada área.
\end{itemize}

\subsubsection{Adjunto}
\begin{itemize}
\item Pertenece únicamente a una subárea.
\item Tendrá una cola de proyectos asignados.
\item Buscará a los expertos especificando el área, la institución, palabras
  clave y valoraciones en las que se prioriza la formalidad de plazos y
  calidad de las evaluaciones. 
\item Una vez finalizada la búsqueda, elegirá a uno o más expertos según
  considere necesario y les enviará un modelo de invitación predeterminada
  mediante correo electrónico para la  evaluación del proyecto.  
\item Avisará de los plazos que tiene el experto para aceptar o declinar una
  invitación según definió el secretario. Además, una vez aceptada dicha
  invitación, le avisará de los plazos de entrega del informe de evaluación,
  también definidos por el secretario. 
\item Podrá reasignar las evaluaciones si el experto declina la invitación,
  no obtiene contestación dentro del plazo o si el experto no cumple con los
  plazos de entrega del informe de evaluación.
\item El adjunto podrá insistir cuando esté próxima la fecha límite de
  entrega del informe de evaluación. 
\item Una vez realizada las evaluaciones de los expertos, el adjunto
  realizará un informe final teniendo en cuenta los informes de los distintos
  expertos que hayan aceptado realizar la evaluación. Este informe final
  deberá ser validado por el coordinador de su área. 
\item Evalúa el trabajo del experto basándose en la formalidad de los plazos
  y la calidad de su informe. 
\item Puede recomendar al secretario añadir expertos.
\item Dentro de la lista de proyectos a las que los adjuntos tienen acceso,
  tendrán una sublista de los expertos que están revisando ese proyecto. 
\item Podrá avisar al coordinador cuando el proyecto no corresponda a su
  subárea. 
\end{itemize}

\subsubsection{Subárea}
\begin{itemize}
\item Sólo podrá pertenecer a un área y tiene un único adjunto asociado.
\end{itemize}

\subsubsection{Expertos}
\begin{itemize}
\item Podrán tener asignados varios proyectos a la vez. 
\item No podrán pertenecer a la misma institución solicitante de la evaluación.
\item Podrá aceptar o declinar las invitaciones de evaluación de proyectos.
\item Cuando acepte la invitación, podrá acceder a la documentación de ese
  proyecto e ir realizando progresivamente el informe en varias sesiones. 
\item Una vez que haya terminado el informe finalizará el proceso de evaluación.
\item Recibirá avisos de finalización de plazos por parte del adjunto para
  finalizar el informe, vía correo electrónico.
\item Cada uno tendrá una serie de palabras clave asociadas a su
  temática. Éstas palabras clave se utilizarán como parámetros en las
  búsquedas. 
\item Tendrá una lista de evaluaciones pendientes, que podrá aceptar o rechazar.
\end{itemize}

\section{Análisis de sistema}
\subsection{Diagrama de flujo de datos}

\imagen{diagramas/DC.pdf}{15.5cm}{Diagrama de Contexto}{dc}
\imagen{diagramas/DFD0.pdf}{15.5cm}{Diagrama de Sistemas}{0}
\imagen{diagramas/DFD1.pdf}{15.5cm}{Gestión de Usuarios}{1}
\imagen{diagramas/DFD2.pdf}{15.5cm}{Gestión de Paquetes Proyectos}{2}
\imagen{diagramas/DFD3.pdf}{15.5cm}{Gestión de Evaluaciones}{3}
\imagen{diagramas/DFD31.pdf}{15.5cm}{Gestión de Modelos}{31}
\imagen{diagramas/DFD32.pdf}{15.5cm}{Gestión de Informes Evaluaciones}{32}
\imagen{diagramas/DFD4.pdf}{15.5cm}{Gestión de Expertos}{4}

\subsection{Definición de flujo de datos}
\subsubsection{Almacenes de datos}
\begin{displaymath}
  \mathbf{USUARIOS} = @id\_usuario + tipo + e-mail + contraseña + nombre +
  apellidos + teléfono + id\_área + institución + currículum +
  palabras\_clave

  \mathbf{PAQUETES PROYECTOS} = @id\_paquete + nombre\_paquete + institución
  + bases + convocatoria + fecha\_entrada + fecha\_salida + fecha\_límite 

  \mathbf{fecha\_entrada} = fecha\_salida = fecha\_límite = día + mes + año

  \mathbf{MODELOS EVALUACIÓN} = @id\_modelo + cppid + estructura

  \mathbf{cppid} = *\emph{Identificador de la convocatoria del paquete de
    proyectos correspondiente}* 

  \mathbf{ÁREAS} = @id\_área + nombre

  \mathbf{SUBÁREAS} = @id\_subárea + @id\_área + id\_usuario + nombre

  \mathbf{SUBÁREAS} = *\emph{El id\_usuario es sólo para los adjuntos. El resto
    de usuarios no tiene área asignada}* 

  \mathbf{PROYECTOS} = @id\_proyecto + @id\_paquete + nombre\_proyecto +
  memoria + id\_área + id\_subárea + estado\_proyecto + tpo\_inv + tpo\_eval
  + id\_modelo\_eval + evaluación 

  \mathbf{tpo\_inv} = *\emph{Tiempo que tiene un experto para aceptar o
    declinar una invitación}* 

  \mathbf{tpo\_eval} = *\emph{Tiempo que tiene un experto para realizar una
    evaluación desde su aceptación}* 

  \mathbf{evaluación} = *\emph{Informe final realizado}*

  \mathbf{EXPERTOS-PROYECTOS} = @id\_usuario + @id\_paqproy + @id\_proyecto +
  id\_informe + valoración + fecha\_asignación 

  \mathbf{valoración} = tpo\_realización + calidad\_informe

  \mathbf{fecha\_asignación} = día + mes + año

  \mathbf{tpo\_realización} = nota

  \mathbf{calidad\_informe} = nota

  \mathbf{INFORMES EVALUACIÓN} = @id\_informe + id\_modelo\_eval +
  estado\_informe + datos 

  \mathbf{datos} = *\emph{El experto, el adjunto y el coordinador rellenarán
    los campos correspondientes al modelo de evaluación del proyecto}* 

  \mathbf{PLANTILLAS CORREOS} = @id\_plantilla + asunto + cuerpo
\end{displaymath}

\subsubsection{Datos Elementales}
\begin{displaymath}
  \mathbf{tipo} = [secretario \mid coordinador \mid adjunto \mid experto]

  \mathbf{dígito} = [0 \mid 1 \mid 2 \mid 3 \mid 4 \mid 5 \mid 6 \mid 7 \mid
  8 \mid 9] 

  \mathbf{día} = [1 \mid 2 \mid 3 \mid 4 \mid 5 \mid 6 \mid 7 \mid 8 \mid 9
  \mid 10 \mid 11 \mid 12 \mid 13 \mid 14 \mid 15 \mid 16 \mid 17 \mid 18
  \mid 19 \mid 20 \mid 21 \mid 22 \mid 23 \mid 24 \mid 25 \mid 26 \mid 27
  \mid 28 \mid 29 \mid 30 \mid 31] 

  \mathbf{mes} = [Enero \mid Febrero \mid Marzo \mid Abril \mid Mayo \mid
  Junio \mid Julio \mid Agosto \mid Septiembre \mid Octubre \mid Noviembre
  \mid Diciembre] 

  \mathbf{nota} = [0 \mid 1 \mid 2 \mid 3 \mid 4 \mid 5 \mid 6 \mid 7 \mid 8
  \mid 9 \mid 10] 

  \mathbf{año} = 2 + dígito + dígito + dígito

  \mathbf{estado\_proyecto} = [sin\_evaluar \mid expertos\_evaluando \mid
  evaluado\_por\_expertos \mid evaluado\_por\_adjunto \mid
  validado\_coodinador] 

  \mathbf{estado\_informe} = [en\_proceso \mid finalizado]

  \mathbf{asunto} = [alta\_usuario \mid invitación \mid aviso\_plazos]
\end{displaymath}

\subsubsection{Flujos de datos}
\begin{displaymath}
  \mathbf{Aceptación} = *\emph{En el perfil del adjunto aparecerá la lista de
    expertos seleccionados y se indica si han aceptado o declinado}* 

  \mathbf{Actualización Estado} = [evaluado\_por\_expertos \mid
  evaluado\_por\_adjunto \mid validado\_coodinador] 

  \mathbf{Adjuntos} = [Consulta Subárea \mid Asignar Subárea]

  \mathbf{Adjuntos} = *\emph{Los usuarios adjuntos añadidos se asignan a una
    subárea. Se escribe en la tabla SUBÁREAS}* 
 
  \mathbf{Alta Usuario} = id\_usuario + tipo + e-mail + contraseña

  \mathbf{Alta Usuario} = *\emph{El secretario da de alta a los usuarios
    introduciendo su identificador, el tipo de usuario, el e-mail y la
    contraseña}* 

  \mathbf{Área} = id\_área + nombre

  \mathbf{Área} = *\emph{Los usuarios añadidos se asignan a una área}*

  \mathbf{Área Asignada} = id\_área

  \mathbf{Área Asignada} = *\emph{Se le añade un área a cada proyecto}*

  \mathbf{Área Proyectos} = id\_área + id\_paqproy + id\_proyecto

  \mathbf{Área Proyectos} = *\emph{Asignación de un área a un proyecto}*

  \mathbf{Asignar Subárea} = id\_subárea + id\_área + id\_usuario +
  nombre\_subárea. 

  \mathbf{Asignar Subárea} = *\emph{Se introduce una fila nueva en la tabla
    SUBÁREAS}* 

  \mathbf{Autentificación} = [Datos Autentificación \mid Verificar Autentificar]

  \mathbf{Autentificación} = *\emph{Acceso al sistema con una cuenta de
    usuario y verificación de la autentificación}* 

  \mathbf{Avisos Plazos} = nombre + apellidos + [tpo\_inv \mid tpo\_eval] +
  id\_proyecto + id\_paquete + nombre\_proyecto + cuerpo 

  \mathbf{Avisos Plazos} = *\emph{Correo de advertencia de fecha límite de
    evaluación de un proyecto}* 

  \mathbf{Cambio Estado} = estado\_proyecto

  \mathbf{Cambio Estado} = *\emph{El estado del proyecto cambia de
    'sin\_evaluar' a 'expertos\_evaluando'}* 

  \mathbf{Consulta Área Proyecto} = área

  \mathbf{Consulta Área Proyecto} = *\emph{Se extrae el área del proyecto
    para buscar expertos relacionados con ese área}* 

  \mathbf{Consulta Autentificación} = e-mail + contraseña

  \mathbf{Consulta Expertos} = [Expertos Seleccionados \mid Expertos Búsqueda
  \mid email] 

  \mathbf{Consulta Informes Expertos} = id\_informe + id\_modelo\_eval +
  estado\_informe + datos 

  \mathbf{Consulta Informes Expertos} = *\emph{Ver informes de proyectos de
    expertos}* 

  \mathbf{Consulta Subárea} = id\_subárea + id\_área + (id\_usuario) + nombre

  \mathbf{Consulta Subárea} = *\emph{Consulta de las subáreas donde trabaja
    el usuario, el id\_usuario es opcional ya que el flujo consultar usuario
    que va desde modificar datos personales a Subáreas no utiliza el
    id\_usuario, sino que lee para asignarle un subárea}* 

  \mathbf{Consulta Usuarios} = id\_usuario + tipo + e-mail + contraseña +
  nombre + apellidos, teléfono + id\_área + institución + currículum +
  palabras\_clave 

  \mathbf{Convocatoria} = convocatoria

  \mathbf{convocatoria}: *\emph{En la convocatoria del paquete de proyectos
    encontramos todo lo necerario para realizar los Modelos de Evaluación}* 

  \mathbf{Correo Alta} = id\_usuario + contraseña + nombre + apellidos + área
  + cuerpo 

  \mathbf{Correo Alta} = *\emph{Correo de confirmación de registro de un
    nuevo usuario en el sistema}* 

  \mathbf{Criterios Evaluación} = [id\_modelo + estructura \mid ID Modelo]

  \mathbf{Criterios Evaluación} = *\emph{Añadir, modificar o eliminar los
    modelos de evaluación a utilizar para realizar los informes}* 

  \mathbf{Datos Autentificación} = id\_usuario + contraseña

  \mathbf{Datos Paquetes Proyectos} = (id\_paquete) + (institución) + (bases)
  + (convocatoria) + (fecha\_entrada) + (fecha\_salida) + (fecha\_límite) 

  \mathbf{Datos Paquetes Proyectos} = *\emph{Se añaden, se modifican y se
    consultan los datos de los paquetes de proyectos}* 

  \mathbf{Datos Personales} = id\_usuario + tipo + e-mail + contraseña +
  nombre + apellidos + teléfono + área + institución + currículum +
  palabras\_clave 

  \mathbf{Datos Personales} = *\emph{Modificación de los datos personales de
    un usuario}* 

  \mathbf{Datos Proyectos} = (id\_proyecto) + (id\_paquete) + (memoria) +
  (área) + (subárea) + (estado\_proyecto) + (tpo\_inv) + (tpo\_eval) +
  (id\_modelo\_eval) + (evaluación) 

  \mathbf{Datos Proyectos} = *\emph{Se añaden, se modifican y se consultan
    los datos de los proyectos. Son todos opcionales por el flujo de datos en
    la modificación}* 

  \mathbf{Datos Usuarios} = id\_usuario
  \mathbf{Datos Usuarios} = *\emph{Se le pasa el id del usuario para
    eliminarlo de la base de datos}* 

  \mathbf{email} = email

  \mathbf{email} = *\emph{Se extrae el email de los expertos para mandarles
    la invitación}* 

  \mathbf{Evaluaciones} = id\_paquete + {evaluación}

  \mathbf{Evaluaciones} = *\emph{Evaluaciones correspondientes a los
    proyectos de un paquete y que son enviadas a la institución convocante
    una vez que estén realizadas}* 

  \mathbf{Experto Desasignado} = id\_usuario + id\_proyecto + id\_paquete

  \mathbf{Expertos} = [Parámetros Consulta \mid Resultado Consulta \mid
  Parámetros Búsqueda \mid Resultado Búsqueda \mid ID Experto \mid Experto
  Desasignado] 

  \mathbf{Expertos} = *\emph{Tráfico relacionado con la búsqueda, consulta,
    asignación y desasignación de expertos para las evaluaciones de
    proyectos}* 

  \mathbf{Expertos Búsqueda} = {@id\_usuario + tipo + e-mail + contraseña +
    nombre + apellidos + teléfono + área + institución + currículum +
    {palabras\_clave}} 

  \mathbf{Expertos Búsqueda} = *\emph{Lista de expertos resultado de la
    consulta}* 

  \mathbf{Expertos Seleccionados} = {@id\_usuario + tipo + e-mail +
    contraseña + nombre + apellidos + teléfono + área + institución +
    currículum + {palabras\_clave}} 

  \mathbf{Expertos Seleccionados} = *\emph{Lista de expertos resultado de la
    consulta}* 

  \mathbf{Expertos-Proyectos} = [Relación Expertos-Proyectos \mid Valoración
  Expertos \mid Fin Relación \mid Fecha Asignación \mid Nueva Relación \mid
  Valoración Trabajo] 

  \mathbf{Fecha Asignación} = fecha\_asignación

  \mathbf{Fecha Asignación} = *\emph{Se utiliza para el aviso de plazos}*

  \mathbf{Fecha Límite} = fecha\_límite

  \mathbf{Fecha Límite} = *\emph{Se utiliza para indicar el plazo de
    realización del informe al experto}* 

  \mathbf{Fin Relación} = id\_usuario + id\_proyecto + id\_paquete

  \mathbf{Fin Relación} = *\emph{Se elimina una tupla de la tabla
    EXPERTOS-PROYECTOS}* 

  \mathbf{ID Experto} = id\_usuario

  \mathbf{ID Modelo} = id\_modelo

  \mathbf{Info Informe Evaluación} = [Informe Rechazado \mid Nuevo Informe]

  \mathbf{Info Paquetes Proyectos} = [Institución Convocante \mid Fecha
  límite \mid Plazo invitación] 

  \mathbf{Información Proyecto} = [Tiempos \mid Consulta Área Proyecto \mid
  Cambio Estado \mid Nombre Proyecto] 

  \mathbf{Informe Final} = id\_proyecto + id\_paquete + estado\_proyecto +
  evaluación 

  \mathbf{Informe Final} = *\emph{Se muestran los datos referidos a los
    informes finales de los proyectos y también el coordinador introduce el
    informe final de evaluación del proyecto una vez finalizado este para ser
    devuelto a la institución convocante}* 

  \mathbf{Informe Rechazado} = id\_informe

  \mathbf{Informe Rechazado} = *\emph{Se elimina la tupla correspondiente en
    la tabla INFORMES EVALUACIÓN}* 

  \mathbf{Informes Adjuntos} = id\_informe + id\_modelo\_eval +
  estado\_informe + datos 

  \mathbf{Informes Adjuntos} = *\emph{Ve y/o modifica el informe final de
    evaluación de un adjunto}* 

  \mathbf{Informes Coordinadores} = id\_informe + id\_modelo\_eval +
  estado\_informe + datos 

  \mathbf{Informes Coordinadores} = *\emph{Validar o modificar informe final
    del adjunto}* 

  \mathbf{Informes Evaluación} = id\_informe + id\_modelo\_eval +
  estado\_informe + datos 

  \mathbf{Informes Evaluación} = *\emph{Introducir, mostrar, modificar o
    eliminar los datos referidos a un informe de evaluación. Además modifica,
    cuando es oportuno, el estado en el que se encuentra dicho informe}* 

  \mathbf{Informes Expertos} = id\_informe + id\_modelo\_eval +
  estado\_informe + datos 

  \mathbf{Informes Expertos} = *\emph{Añadir o modificar informes de
    proyectos por parte de los expertos}* 

  \mathbf{Institución Convocante} = institución

  \mathbf{Institución Convocante} = *\emph{Se consulta la institución
    convocante para que no coincida en la búsqueda de expertos}* 

  \mathbf{Invitación} = [Petición \mid Aceptación]

  \mathbf{Invitación} = *\emph{Correo de invitación a un experto para que
    evalúe un proyecto y respuesta de éste}* 

  \mathbf{Lista Usuarios} = {id\_usuario + tipo + e-mail + contraseña +
    nombre + apellidos + teléfono + área + institución + currículum +
    {palabras\_clave}} 

  \mathbf{Lista Usuarios} = *\emph{Lista devuelta por una búsqueda o consulta
    de usuarios}* 

  \mathbf{Memoria Proyecto} = id\_proyecto + id\_paquete + memoria +
  id\_modelo\_eval + estado\_proyecto 

  \mathbf{Memoria Proyecto} = *\emph{Se lee para poder realizar el
    informe. El adjunto también leerá el estado en que se encuentra la
    evaluación del proyecto}* 

  \mathbf{Modelo} = id\_modelo\_eval

  \mathbf{Modelo} = *\emph{Se consulta el id del modelo de evaluación para
    poder asignárselo a los proyectos y también se consulta por si se quiere
    hacer un cambio en paquetes de proyectos}* 

  \mathbf{Modelos Evaluación} = [id\_modelo + estructura \mid Nuevo Modelo
  Evaluación \mid id\_modelo] 

  \mathbf{Modelos Evaluación} = *\emph{Insertar, modificar o eliminar las
    estructuras que deben seguir los informes de evaluación de cada
    proyecto}* 

  \mathbf{Nombre Proyecto} = id\_proyecto + nombre

  \mathbf{Nueva Relación} = id\_usuario + id\_proyecto + id\_paquete +
  id\_informe + valoración + fecha\_asignación 

  \mathbf{Nueva Relación} = *\emph{Se inserta una tupla en la tabla
    EXPERTOS-PROYECTOS}* 

  \mathbf{Nuevo Informe} = id\_informe + id\_modelo\_eval + estado\_informe +
  datos 

  \mathbf{Nuevo Informe} = *\emph{Se inserta una tupla en la tabla INFORMES
    EVALUACIÓN. El estado\_informe será 'en\_proceso'}* 

  \mathbf{Nuevo Modelo Evaluación} = id\_modelo + estructura

  \mathbf{Nuevo Modelo Evaluación} = *\emph{Añade un nuevo Modelo de
    Evaluación}* 

  \mathbf{Paquetes Proyectos} = [id\_paquete + institución +
  nº\_de\_proyectos + bases + convocatoria + fecha\_entrada + fecha\_salida +
  fecha\_límite \mid Proyectos] 

  \mathbf{Paquetes Proyectos} = *\emph{Paquetes de proyectos enviados por la
    institución convocante para su evaluación}* 

  \mathbf{Parámetros} = (id\_usuario) + (tipo) + (e-mail) + (nombre) +
  (apellidos) + (teléfono) + (área) + (institución) + (currículum) +
  (palabras\_clave) 

  \mathbf{Parámetros} = *\emph{Son los parámetros de búsqueda de
    usuarios. Todos los campos son opcionales ya que se pueden realizar
    búsquedas de usuarios de varias maneras}* 

  \mathbf{Parámetros Consulta} = Parámetros Búsqueda = (id\_usuario) +
  (tipo) + (e-mail) + (contraseña) + (nombre) + (apellidos) + (teléfono) +
  (área) + (institución) + (currículum) + (palabras\_clave) 

  \mathbf{Petición} = id\_usuario + nombre + apellidos + id\_proyecto +
  nombre\_proyecto + tmp\_invitación + Correo\_invitación 

  \mathbf{Plantilla} = id\_plantilla + asunto + cuerpo.

  \mathbf{Plantilla} = *\emph{Para mandar los correos pertinentes}*

  \mathbf{Plazo Invitación} = tmp\_invitación

  \mathbf{Proyecto Rechazado} = id\_proyecto + id\_paquete + id\_subárea

  \mathbf{Proyecto Rechazado} = *\emph{Aviso del adjunto al coordinador de su
    área para informar de que el proyecto no corresponde a su subárea y debe,
    por tanto, asignarse a otro subárea distinta}* 

  \mathbf{Proyectos} = id\_proyecto + id\_paquete + nombre\_proyecto +
  memoria + área + subárea + estado\_proyecto + tpo\_inv + tpo\_eval +
  id\_modelo\_eval + evaluación 

  \mathbf{Proyectos} = *\emph{Consulta la memoria del proyecto para realizar
    el informe de evaluación}* 

  \mathbf{Relación Expertos-Proyectos} = {id\_usuario + id\_proyecto +
    id\_paquete + id\_informe + valoración + fecha\_asignación} 

  \mathbf{Relación Expertos-Proyectos} = *\emph{Lista de la relación entre
    Expertos y Proyectos. Se devolverá una lista según los parámetros de
    consulta}* 

  \mathbf{Resultado Consulta} = Resultado Búsqueda = {id\_usuario + tipo +
    e-mail + contraseña + nombre + apellidos + teléfono + área + institución
    + currículum + {palabras\_clave}} 

  \mathbf{Subárea Adjuntos} = id\_subárea + id\_usuario

  \mathbf{Subárea Adjuntos} = *\emph{Asignación de un adjunto a una subárea}*

  \mathbf{Subárea Asignada} =id\_subárea

  \mathbf{Subárea Asignada} = *\emph{Se le añade una subárea a cada proyecto}*

  \mathbf{Subárea Proyectos} = id\_subárea + id\_proyecto + id\_paquete

  \mathbf{Subárea Proyectos} = *\emph{Asignación de un proyecto a una subárea}*

  \mathbf{Tiempos} = tpo\_eval

  \mathbf{Tiempos} = *\emph{Se consulta el tiempo de evaluación para saber
    cuanto le queda al experto para realizar el informe}* 

  \mathbf{Usuarios} = [id\_usuario + tipo + e-mail + contraseña + (nombre) +
  (apellidos) + (teléfono) + (área) + (institución) + (currículum) +
  (palabras\_clave) \mid Parámetros] 

  \mathbf{Usuarios} = *\emph{Da el alta a los usuarios, los consulta o
    modifica sus datos. Modifica el coordinador asociado a un área. Incluye
    los parámetros de búsqueda introducidos para realizar dicha búsqueda de
    usuarios}* 

  \mathbf{Valoración Expertos} = id\_usuario + id\_proyecto + id\_paquete +
  tpo\_realización + calidad\_informe 

  \mathbf{Valoración Expertos} = *\emph{Cada informe elaborado por un experto
    para un proyecto, recibirá una valoración según el tiempo de realización
    y la calidad del informe}* 

  \mathbf{Valoración Trabajo} = {tpo\_realización + calidad\_informe}

  \mathbf{Valoración Trabajo} = *\emph{Se utiliza en búsqueda de expertos}*

  \mathbf{Verificar Autentificar} = *\emph{Mostrar que la autentificación es
    correcta. Si introduce incorrectamente el id o la contraseña o ambas, se
    mostrará una plantalla en la que se indicará: 'Usuario o contraseña son
    incorrectas'}* 
\end{displaymath}

\section{Diseño del sistema}
\subsection{Diagrama de diseño estructurado}
\imagen{diagramas/DE_1.pdf}{15cm}{\small{Gestión distribuida de la revisión de Proyectos de Investigación}}{} 
\imagen{diagramas/DE_2.pdf}{15cm}{\small{Gestión de Usuarios}}{}
\imagen{diagramas/DE_3.pdf}{15cm}{\small{Gestión de Paquetes de Proyectos}}{}
\imagen{diagramas/DE_4.pdf}{15cm}{\small{Gestión de Evaluaciones}}{}
\imagen{diagramas/DE_5.pdf}{15cm}{\small{Gestión de modelos}}{}
\imagen{diagramas/DE_6.pdf}{14cm}{\small{Gestión de Informes de Evaluaciones}}{}
\imagen{diagramas/DE_7.pdf}{14cm}{\small{Gestión de Expertos}}{}

\section{Base de datos}
\subsection{Diagrama Entidad/Interrelación}
\imagen{diagramas/ER.pdf}{15cm}{Diagrama Entidad/Interrelación}{}

\subsection{Diagrama relacional}
\imagen{diagramas/MRE.pdf}{15cm}{Diagrama Relacional}{}

\section{Manual de usuario}
\input{manual}

\section{Implementación}
\subsection{Introducción}

\subsection{Código fuente}
\subsubsection{index.php}
\lstinputlisting{implementacion/index.php}

\subsubsection{settings.php}
\lstinputlisting{implementacion/settings.php}

\subsubsection{js/evalmodels.js}
\lstinputlisting{implementacion/js/evalmodels.js}

\subsubsection{js/evalreports.js}
\lstinputlisting{implementacion/js/evalreports.js}

\subsubsection{js/main.js}
\lstinputlisting{implementacion/js/main.js}

\subsubsection{js/projects.js}
\lstinputlisting{implementacion/js/projects.js}

\subsubsection{source/evaluation\_models.php}
\lstinputlisting{implementacion/source/evaluation_models.php}

\subsubsection{source/evaluation\_reports.php}
\lstinputlisting{implementacion/source/evaluation_reports.php}

\subsubsection{source/loginout.php}
\lstinputlisting{implementacion/source/loginout.php}

\subsubsection{source/mysql.php}
\lstinputlisting{implementacion/source/mysql.php}

\subsubsection{theme/evaluation\_models.php}
\lstinputlisting{implementacion/theme/evaluation_models.php}

\subsubsection{theme/evaluation\_reports.php}
\lstinputlisting{implementacion/theme/evaluation_reports.php}

\subsubsection{theme/login.php}
\lstinputlisting{implementacion/theme/login.php}

\subsubsection{theme/main.php}
\lstinputlisting{implementacion/theme/main.php}

\subsubsection{theme/projects.php}
\lstinputlisting{implementacion/theme/projects.php}

\subsubsection{theme/users.php}
\lstinputlisting{implementacion/theme/users.php}

\subsubsection{theme/css/dialog.php}
\lstinputlisting{implementacion/theme/css/dialog.php}

\subsubsection{theme/css/evalmodels.css}
\lstinputlisting{implementacion/theme/css/evalmodels.css}

\subsubsection{theme/css/evalreports.css}
\lstinputlisting{implementacion/theme/css/evalreports.css}

\subsubsection{theme/css/login.css}
\lstinputlisting{implementacion/theme/css/login.css}

\subsubsection{theme/css/main.css}
\lstinputlisting{implementacion/theme/css/main.css}

\subsubsection{theme/css/users.php}
\lstinputlisting{implementacion/theme/css/users.php}

\section{Carga de trabajo}
\begin{center}
 \begin{tabular}{|p{10cm}|c|}\hline 
  Apellidos y Nombre & Porcentaje \\ \hline \hline
  de la Rubia García-Carpintero, Sergio & 20\% \\ \hline
  Millán Sánchez-Grande, Miguel & 20\% \\ \hline
  Muñoz Villarreal, Luis & 20\% \\ \hline
  Serrano Sánchez, Alicia & 20\% \\ \hline
  Torres Triviño, Juan Miguel & 20\% \\ \hline 
 \end{tabular}
\end{center}

\section{Bibliografía}

\end{document}
